\chapter{Discussion}



- trying to relax assumptions, however important to note that using US and e.g. elastography introduces a lot of other assumptions. So there still is a long way to a completely realistic model. ...
    - Finni - Finni (2006): During a contraction, the fibers in a pennate muscle rotate and there is shear strain in the interface between the muscle fibers (Gans, 1982; Huijing, 1999). 
    -> US imaging only 2D plane, so out of plane fibre rotation is not taken into account. 

-  "Disregarding the 3D conformation of tendon introduces a systematic underestimation of tendon length and thus an overestimation of length changes." \cite{seynnes_ultrasound-based_2014}.


% copied
\subsubsection{elastography}
- using elastography there is variation due to handling of probe, and different technical settings \cite{kot_elastic_2012}
	first e.g. probe mounts have to be developed, and the influence on the quantitative nature of the various  elastograpghy methods has to be further studied, before it is applicable to sys ID. Exact protocols have to be developed...

- abundance of studies uses SSI to assess youngs modulus (e.g. [shinohara 2010, Arda 2011, maisetti 2012], ...see Brandenburg et al. 2014), however the quantitative nature of this measurement cannot be validated in vivo. The in vitro study of \cite{eby_validation_2013} validated this, but did not found the relation $E=3\mu$, but approximately $E=5\mu$. This difference indicates that it might be a more qualitative measurement, but since E is assumed to increase as function of the group velocity squared, the influence on the error in estimation might be limited... 

- for elastography compound imaging can be used to increase image quality, but alternative beamforming methods (Garcia 2013, Montaldo 2009) /RF remapping (check Sumanaweera 2005) can also be used to improve image quality, without decreasing effective frame rate %\cite{cortes_continuous_2015}
% end copied



\subsubsection{MTU interaction and afferent feedback}
In many studies, the muscle tendon unit interaction is omitted, or other methods to estimate the muscle and tendon elongation are used. EXAMPLES, studies that estimate using ankle angle… Not the way to go. The total elongation of the MTU can still be used, and the additional information of US can be used to quantify the individual contribution of lengthening, since one elongation of the (series) system is known. 


