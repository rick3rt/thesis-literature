\chapter{Conclusion}

\begin{itemize}
	\item Developments in ultrafast ultrasound have resulted in various modalities that can be for various clinical applications. In the context of neuromuscular system identification two useful modalities were found, being \textbf{(1)} shear wave elastography and \textbf{(2)} ultrafast imaging of muscle-tendon units.

	\item In the models using in parametric neuromuscular system identification six main assumptions were identified, relating to \textbf{(1)} lumping of antagonistic muscle groups, \textbf{(2)} whether or not muscle-tendon unit interaction was considered, \textbf{(3)} modelling of muscle activation, \textbf{(4)} linear versus nonlinear modelling, \textbf{(5)} considered sources of afferent feedback and \textbf{(6)} the modelling of intrinsic feedback. 

	\item It can be concluded that before elastography measurements can be of value in system identification, the currently available shear wave elastography techniques require extensive validation. Conflicting reports on assessed material parameters and conflicting reports on validity of employed methodologies show the need for further developments and validation in the field of elastography before it can be employed in system identification. 
	
	\item Current reports on ultrafast imaging of muscle-tendon units (MTU) show promising results for the application of ultrafast imaging during system identification experiments. It can be used to relax the assumptions of unknown MTU interactions, obtaining sources of afferent activity and de-lumping of antagonistic muscles.
	
\end{itemize}

