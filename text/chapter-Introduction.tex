\chapter{Introduction}
%To gain a better understanding of the pathophysiology of these movement disorders, many studies have been conducted to identify the contribution of the various physiological mechanisms to motor behaviour. The goal of these studies is to build an accurate computational neuromuscular model, which can be of great value in identifying the structural or neural cause of a movement disorder, can be used to predict the effects of interventions and can aid in developing adequate treatment \cite{meskers_neurocontrol_2015}.

The human neuromuscular system has various mechanisms to modulate movement and reflexes, which can be separated in different modules. The individual contribution of these mechanisms to movement during various motor tasks is not exactly known. In literature, two main contributors to movement modulation have been described, instantaneous intrinsic stiffness (coming from voluntary (co)contraction and tissue viscoelastic properties) and reflexive stiffness (evoked by afferent proprioceptive feedback). System identification has been applied to address the closed-loop interactions between these different (sub)modules, and identify their contribution to movement. The ultimate goal is to build a computational neuromuscular model, which can be used to gain a better understanding of the neuromuscular system. The identification of the neuromuscular system is not straightforward, due to the fact that the contribution of voluntary contractions, reflexes and intrinsic properties are hard to separate. Internal signals cannot be measured noninvasively using conventional methods, which hampers the identification process. Having an accurate neuromuscular model can be of use in various scientific fields. It can help in identifying the cause of certain movement disorders \cite{meskers_neurocontrol_2015}, in the development of active prosthetics \cite{eilenberg_control_2010, markowitz_speed_2011, eilenberg_development_2018} and in developing human-machine interfaces \cite{hosseini_neuromuscular_2010}. 
%Having an accurate neuromuscular model can help in identifying the cause of certain movement disorders. Furthermore, several engineering fields could benefit from such a model, e.g. the development of active prosthetics and human-machine interfaces. 

%
%The underlying physiological mechanisms during motor control in various tasks remains elusive, due to many unknown interactions between these mechanisms and 
%
%. The contribution of the many different physiological mechanisms to motor control in various tasks is not well understood. 
%
%System identification has been applied to address the closed-loop interactions between different (sub)modules, and identify the contribution of these mechanisms to movement and reflexes. The goal is to build a computational neuromuscular model, which can be used to gain a better understanding of the neuromuscular system, and subsequently identify the cause of certain movement disorders. Furthermore, several engineering fields could benefit from such a model, e.g. the development of active prosthetics and human-machine interfaces. 

% Having an adequate model can be of great value in identifying the structural or neural cause of a movement disorder, can be used to predict the effects of interventions and can aid in developing adequate treatment \cite{meskers_neurocontrol_2015}. 


\subsection*{Movement disorders}
%\tred[more general, wide variety of movement disorders, from which it is not possible to see the cause. Internal mechanisms, high stiffness tissue components can be found, but cause of change remains elusive. Distinction between neural or structural cause not possible. consequently no accurate treatment, since only symptoms can be treated, and not the cause.]

The pathophysiology of neurological movement disorders arising after trauma or caused by neurological diseases is poorly understood. A wide variety of neurological diseases, such as stroke, cerebral palsy, multiple sclerosis and complex regional pain syndrome, often lead to movement disorders, among which dystonia, myoclonus, tremor and spastic paresis. The symptoms of these movement disorders can easily be observed from the distorted motor patterns, characterized by poor coordination, sustained contractions, abnormal fixed postures, jerky movements or sudden loss of muscle tone, depending on the type of disorder \cite{levy_myoclonus_2016, de_gooijer-van_de_groep_differentiation_2013, munts_fixed_2011}. On the other hand, the cause of these movement disorders remains elusive. The symptoms can be caused by either neural, e.g. improper muscle activation, or structural abnormalities, e.g. altered viscoelastic properties of tissue \cite{de_gooijer-van_de_groep_differentiation_2013}. To gain a better understanding of the pathophysiology of these movement disorders, many studies have been conducted to identify the contribution of the various physiological mechanisms to motor behaviour of healthy subjects. The goal of these studies is to build an accurate computational neuromuscular model, which can be of great value in identifying the structural or neural cause of a movement disorder. In addition, it is expected that an accurate neuromuscular model is useful to predict the effects of interventions and can aid in developing adequate treatment \cite{meskers_neurocontrol_2015}.


%Complex regional pain syndrome (CRPS) is a painful disorder that can occur after trauma or spontaneously, with a broad spectrum of symptoms relating to sensory, autonomic and motor impairments \cite{mugge_reflex_2011}. There are several movement disorders associated with CRPS, being dystonia, myoclonus and tremor. Fixed dystonia is characterized by sustained muscle contractions, resulting in abnormal fixed postures. Symptoms of myoclonus are abrupt jerky, shock-like movements caused by sudden contractions or sudden loss of muscle tone, which may occur in sequence, in a pattern or random \cite{levy_myoclonus_2016}. Tremor is associated with movement disorders that cause involuntary, oscillatory and rhythmic movement of body parts. 
%Other types of movement disorders caused by upper motor neuron lesions are multiple sclerosis, cerebral palsy and stroke, and are also poorly understood, hampering adequate treatment. 

% The pathophysiology of movement disorders caused by CRPS or upper motor neuron lesions (such as stroke, multiple sclerosis/ and cerebral palsy) is poorly understood, hampering adequate treatment. 
% other neurological impairments such as multiple sclerosis (), cerebral palsy (characterised by poor coordination, stiff and/or weak muscles and tremors) and ]

\subsection*{Active prosthetics}
Another field that could benefit from an adequate model of the human neuromuscular system, is the field concerned with the development of active prosthetics. Approximately 90\% of new amputations concern the lower extremity, of which in more than half the cases the patient is older than 65 years \cite{windrich_active_2016}. Passive prosthetics can be used for transtibial amputees to regain basic standing and walking functionality. However, it is known that during gait at moderate to fast walking speeds the human ankle generates net positive mechanical work \cite{gates_characterizing_2004}. Available energy storage and return prosthetics are able to store energy in the springs during the stance phase and release this during push off, conserving part of the energy, but cannot deliver the additional required work. As a consequence, patients have to modify their motor patterns, e.g. by generating the missing work by compensating with the hip \cite{winter_biomechanics_1988}, resulting in an increased oxygen consumption of approximately 20\% during walking at all speeds, compared to nonamputees \cite{molen_energy/speed_1973}. 

To generate the missing mechanical work to mimic `normal' walking, active prosthetics have been developed, but there are still many control challenges for walking under various conditions, e.g. varying speed and terrain \cite{markowitz_speed_2011}. The human neuromuscular system has no problems in adapting to these varying conditions, and therefore multiple studies, e.g. \cite{eilenberg_control_2010, markowitz_speed_2011, eilenberg_development_2018}, have tried to incorporate a neuromuscular model in the control of active prosthetics, enabling emulation of biological reflexes. This allows the controller to automatically adapt to different conditions, e.g. increased slope or speed, without the need for manually changing the controller settings. 
% \citeauthor{markowitz_speed_2011} proposed using a neuromuscular model, to emulate biological reflexes, and provide closed-loop feedback control to develop a speed adaptive controller that generates the required mechanical work to resemble nonamputee walking \cite{markowitz_speed_2011}. 

\subsection*{Human-machine interfaces}
Lastly, the design of human-machine interfaces could benefit from a neuromuscular model. Manual control tasks are prone to human errors and therefore there is the desire to fully automate (sub)tasks, or to provide the human with sufficient alerting systems to make less errors. However, fully automating complex tasks is in many cases not possible, or changing the role of the human to supervisor is undesirable \cite{sheridan_humans_2002}. The field of haptic shared control deals with combining the control of a possibly imperfect automation system, and the control of the human, by letting them both exert forces on the system to control \cite{hosseini_neuromuscular_2010}. The design of the feedback forces exerted by the controller is a challenge in the development of haptic shared control systems. Due to the lack of quantitative knowledge about the response of the neuromuscular system to forces, the design of such `shared' control systems is in many cases a trial-and-error process, often resulting in systems where subjects feel out of control. It is expected that this design process can benefit from an accurate neuromuscular model, allowing optimisation of the feedback forces to create a natural feeling of shared control \cite{hosseini_neuromuscular_2010}. 

\subsection*{Neuromuscular modelling}
The development of an accurate neuromuscular model describing the joint dynamics faces many challenges, caused by the complexity of the human neuromuscular system and its ability to adapt to various tasks and conditions. There are two principal mechanisms that allow modulation of joint dynamics, being intrinsic and afferent (i.e. reflexive) feedback. These mechanisms are able to change the mechanical admittance of the joint, defined as the dynamic behaviour (i.e. movement) in response to force perturbations \cite{schouten_nmclab_2008}. To resist perturbations, the mechanical admittance of the joint has to be reduced, which can be done by (voluntary) changing the intrinsic properties (i.e. muscle co-contraction) or can be (involuntary) adapted by reflexive muscle activation. 

Intrinsic feedback is instantaneous and arises from the mechanical properties of both passive tissue and (co-)contracted muscles. The main source of intrinsic feedback arises from the viscoelastic properties of the muscle, determined by the muscle’s force-length and force-velocity characteristics \cite{winter_biomechanics_1988}. Increased \mbox{(co-)contraction} of muscles enlarges the muscles' viscoelasticity, and hence its ability to resist perturbations (i.e. low admittance). 

On the other hand, reflexive muscle activation is able to modulate the joint admittance, evoked by afferent feedback from sensory organs in the muscle, the proprioceptors. Physiological studies have identified two main proprioceptors, the muscle spindles (MS), providing muscle stretch and stretch velocity feedback, and Golgi tendon organs (GTO), providing muscle force feedback. The way in which these sources afferent feedback are used to modulate the joint admittance in response to perturbations is not well understood and often debated \cite{loeb_hard_1987}. Many physiological and experimental studies have been conducted to identify the individual contribution of the different afferent sources during motion, but many nested feedback loops hamper the identification process \cite{schouten_nmclab_2008}. 

The identification of the individual contribution of MS and GTO feedback and intrinsic properties to joint admittance modulation is further complicated by limitations in measuring internal signals such as muscle and tendon elongation, muscle force and neural activity \cite{allen_why_2016}. Due to significant muscle-tendon interaction, joint rotation is not always proportional to muscle elongation, and consequently the contribution of MS feedback to reflexes is harder to estimate \cite{zajac_muscle_1989}. Several model-based methods have been proposed to estimate individual muscle force, but validation of the models is hard, due to the redundancy of the muscular system, and the qualitative nature of electromyography (EMG) measurements \cite{erdemir_model-based_2007}. 

The wide variety of models of the neuromuscular system that have been created over the last couple decades, have been made by introducing a lot of assumptions, to make the system identification problem manageable. In general, joint identification experiments are conducted by letting a subject reject perturbations generated by a robotic manipulator. Using data such as recorded joint angle, torque exerted on manipulator and EMG muscle activity measurements, the parameters in the proposed (simplified) model structure can be estimated. Although it is known that the neuromuscular system behaves highly nonlinear, in general the used model structures are linear, and linear identification techniques are used \cite{van_der_helm_identification_2002, schouten_nmclab_2008, mugge_rigorous_2010}. As a consequence, low amplitude perturbations have to be used, and only the joint dynamics near the perturbed operating point are identified. Furthermore, muscle-tendon interactions are often not modelled, and muscle elongation is taken proportional to joint deflection, implying an infinitely stiff tendon \cite{zhang_simultaneous_1997, mirbagheri_intrinsic_2000, van_der_helm_identification_2002}. Due to the poor understanding and ongoing debate about the physiology of the proprioceptors in reflexes, there is a lot of variation in the modelling of reflexive feedback. Often, not even individual muscles are modelled, but both flexor and extensor muscles are lumped as one system, thus not considering moment arms of the individual muscles in the joint. 
% ADD MORE LINEAR, AND OTHER STUDIES TO RIGHT PLACE XXX

\subsection*{Ultrafast ultrasound}
Although the models created under these assumptions and with the available data show promising results, further improvements in our understanding of the neuromuscular system, and thus our ability to model the system, require additional measurements of internal signals. To gain a better understanding of the reflexive mechanisms, measuring the force in the muscle and the elongation of the muscle fibres is of great importance in separating contributions of MS and GTO activity. Different available medical imaging techniques allow imaging of the musculoskeletal system, however, their usability in joint dynamic system identification is limited, due to the required high temporal resolution (e.g. monosynaptic reflexes stimulus to response within 50-90 \si{\milli\second} \cite{latash_6_2016}). Imaging techniques, such as MRI and X-ray, have either a too low temporal resolution, are costly, require large imaging devices or expose subjects to undesired radiation. Ultrasound is a much more flexible imaging technique with a much higher temporal resolution that has already been used in various experiments to image muscle-tendon interaction, e.g. \cite{klint_afferent_2009, cronin_automatic_2011}. These studies are mostly using conventional ultrasound, which allows imaging up to \SI{60}{\hertz} (depending on acoustic travel time, decreasing with imaging depth) \cite{minin_ultrafast_2011}. Although this allows visualising some of the muscle-tendon interaction, to further improve the understanding of reflexes, higher frame rates are desired to have high frequent measurements of muscle fibre and tendon elongation. 

A new disruptive imaging technology that has been long studied in research, but only recently made its entrance in clinical applications due to recent developments in graphical processing units (GPU), is ultrafast ultrasound \cite{tanter_ultrafast_2014}. Instead of using multiple focused waves to construct an image, like in conventional ultrasound, it uses a single plane wave to insonify the complete region of interest in the object to image \cite{tanter_ultrafast_2014}. Next, the image can be reconstructed from the recorded backscattered echoes by performing digital parallel beamforming, which is a computationally expensive process. This technique tremendously increases the frame rate, allowing imaging at frame rates of up to \SI{15000}{\hertz} (for superficial tissue) \cite{minin_ultrafast_2011}. These ultrafast frame rates open up new possibilities in e.g. the visualisation of transient shear waves, allowing the field of elastography to quantitatively determine the (visco)elastic properties of tissue \cite{gennisson_ultrasound_2013}. 

\subsection*{Goal}
The high frame rate of ultrafast ultrasound imaging allows doing new non-invasive muscle fibre and tendon elongation measurements during joint dynamics system identification experiments. Furthermore, ultrafast imaging in combination with elastography could potentially be used for quantitative measurements of the viscoelastic properties of muscle and tendon. This raises the question: 
\begin{displayquote}
Which assumptions in joint dynamics system identification can be relaxed by using Ultrafast Ultrasound?
\end{displayquote}

In this literature survey, first the working principle of ultrafast ultrasound will be explained in \autoref{chap:ultrafast_ultrasound}. In \autoref{chap:assumptions}, the assumptions made in modelling of the neuromuscular system will be identified. Finally, in chapter \ref{chap:remove_assumptions}, the possibilities of using ultrafast ultrasound in neuromuscular system identification will be discussed, and a view on the assumptions that can be relaxed will be presented. %In what way ultrafast ultrasound can help in removing the assumptions and simplifications, and consequently help in creating more realistic neuromuscular models.


% removed 25-Oct-18; 00:36, updated with piece above
% The pathophysiology of neurological movement disorders arising after trauma or caused by neurological diseases is poorly understood. Complex regional pain syndrome (CRPS) is a painful disorder that that can occur after trauma or spontaneously, with a broad spectrum of symptoms relating to sensory, autonomic and motor impairments \cite{mugge_reflex_2011}. There are several movement disorders associated with CRPS, being dystonia, myoclonus and tremor. Fixed dystonia is characterized by sustained muscle contractions, resulting in abnormal fixed postures. Symptoms of myoclonus are abrupt jerky, shock-like movements caused by sudden contractions or sudden loss of muscle tone, which may occur in sequence, in a pattern or random \cite{levy_myoclonus_2016}. Tremor is associated with movement disorders that cause involuntary, oscillatory and rhythmic movement of body parts. 

% Another field that could benefit from an adequate model of the human neuromuscular system, is the development of active prosthetics. Approximately 90\% of the new amputations concern the lower extremity, of which more than half is older than 65 years \cite{windrich_active_2016}. Passive prosthetics can be used for transtibial amputees to regain basic standing and walking functionality. However, it is known that during gait at moderate to fast walking speeds the human ankle generates net positive mechanical work \cite{gates_characterizing_2004}. Available energy storage and return prosthetics are able to store energy in the springs during the stance phase and release this during push off, but cannot deliver the additional required work. As a consequence, patients have to modify their motor patterns, e.g. compensate with the hip \cite{winter_biomechanics_1988}, resulting in increased oxygen consumption of about 20\% during walking at similar speeds compared to nonamputees \cite{molen_energy/speed_1973}. To generate the required additional mechanical work to resemble ‘normal’ walking, active prosthetics have been developed, but there are still many control challenges for walking under various conditions, e.g. varying speed and terrain \cite{markowitz_speed_2011}. \citeauthor{markowitz_speed_2011} proposed using a neuromuscular model, to emulate biological reflexes, and provide closed-loop feedback control to develop a speed adaptive controller that generates the required mechanical work to resemble nonamputee walking \cite{markowitz_speed_2011}. 

% Lastly, the design of human-machine interaction could benefit from a neuromuscular model. Manual control tasks are prone to human errors and therefore there is the desire to fully automate (sub)tasks, or to provide the human with sufficient alerting systems to make less errors. However, fully automating complex tasks is very often not possible, or changing the task of the human to supervisor is undesirable. The field of haptic shared control deals with combining the control of an (possibly imperfect) automation system, and the control of the human, by letting them both exert forces on the system to control \cite{hosseini_neuromuscular_2010}. The design of the feedback forces exerted by the controller is a challenge in the development of haptic shared control systems, and due to the lack of quantitative knowledge of neuromuscular response to forces, the design of such systems is often a trial-and-error process. It is expected that this design process can benefit from an accurate neuromuscular model \cite{hosseini_neuromuscular_2010}. 

% The development of an accurate neuromuscular model describing the dynamics of a joint faces many challenges, caused by the complexity of the human neuromuscular system and its adaptability to various tasks and conditions. There are two principal mechanisms that allow modulation of the dynamics of a joint, being intrinsic and afferent (i.e. reflexive) feedback. These mechanisms are able to change the mechanical admittance of the joint, defined as the dynamic behaviour (i.e. movement) in response to force perturbations \cite{schouten_nmclab_2008}. To resist perturbations, the mechanical admittance of the joint has to be reduced, which can be done by (voluntary) changing the intrinsic properties (i.e. muscle co-contraction) or can be evoked by reflexive muscle activation. 

% Intrinsic feedback is instantaneous and arises from the mechanical properties of both passive tissue and (co-)contracted muscles. The main source of intrinsic feedback is seen as the viscoelastic properties of the muscle, determined by the muscle’s force-length and force-velocity characteristics \cite{winter_biomechanics_1988}. Increased (co-)contraction of muscles enlarges the muscles’ viscoelasticity, and hence its ability to resist perturbations (i.e. low admittance). 

% On the other hand, reflexive muscle activation is able to modulate the joint admittance, evoked by afferent feedback from sensory organs in the muscles (proprioceptors). Physiological studies have identified two main proprioceptors, the muscle spindles (MS), providing muscle stretch and stretch velocity feedback, and Golgi tendon organs (GTO), providing muscle force feedback. The way in which this afferent feedback is used to modulate the joint admittance in response to perturbations is not well understood and often debated \cite{loeb_hard_1987}. Many physiological and experimental studies have been conducted to identify the individual contribution of the different afferent sources during motion, but many nested feedback loops hamper the identification process \cite{schouten_nmclab_2008}.

% The identification of the individual contribution of MS and GTO feedback and intrinsic properties to joint admittance modulation is further complicated by limitations in measuring internal signals such as muscle and tendon elongation, muscle force and neural activity \cite{allen_why_2016}. Due to significant muscle-tendon interaction, joint rotation is not always proportional to muscle elongation, and consequently the contribution of MS feedback to reflexes are harder to estimate \cite{zajac_muscle_1989}. Several model-based method have been proposed to estimate individual muscle force, but validation of the models is hard, due to the redundancy of the muscular system, and the qualitative nature of electromyography (EMG) measurements \cite{erdemir_model-based_2007}

% The various models of the neuromuscular system that have been created over the last couple decades, have been made by introducing a lot of assumptions, to make the system identification problem manageable. In general, joint identification experiments are conducted by letting a subject reject perturbations generated by a robotic manipulator. Using data such as recorded joint angle, torque exerted on manipulator and EMG measurements, the parameters in the proposed (simplified) model structure can be estimated. Although it is known that the neuromuscular system behaves highly nonlinear, in general the used model structures are linear, and linear identification techniques are used \cite{ van_der_helm_identification_2002, schouten_nmclab_2008, mugge_rigorous_2010}. As a consequence, low amplitude perturbations have to be used, and only the joint dynamics near the perturbed operating point are identified. Furthermore, muscle-tendon interactions are often not modelled, and muscle elongation is taken proportional to joint deflection, implying an infinitely stiff tendon \cite{zhang_simultaneous_1997, mirbagheri_intrinsic_2000, van_der_helm_identification_2002}. Due to the poor understanding and ongoing debate about the physiology of the proprioceptors in reflexes, there is a lot of variation in the modelling of reflexive feedback. Often, not even individual muscles are modelled, but both flexor and extensor muscles are lumped as one system, thus not considering moment arms of the individual muscles in the joint. 

% Although the models created under these assumptions and with the available data show promising results, to further improve our understanding of the neuromuscular system, and thus our ability to model the system, additional measurements of internal signals are required. To gain a better understanding of the reflexive mechanisms, measuring the force in the muscle and the elongation of the muscle fibres is of great importance in separating contributions of MS and GTO activity. Different available medical imaging techniques allow imaging of the musculoskeletal system, however their usability in joint dynamic system identification is limited, due to the high required temporal resolution (monosynaptic reflexes stimulus to response within 50-90 \si{\milli\second} \cite{latash_6_2016}). Imaging techniques such as MRI and X-ray, have either a too low temporal resolution, are expensive, require large imaging devices or expose subjects to radiation. Ultrasound is a much more flexible image technique with a much higher temporal resolution that has already been used in various experiments to image muscle-tendon interaction, e.g. \cite{af_klint_afferent_2009, cronin_automatic_2011}. These studies are mostly using conventional ultrasound, which allows imaging up to \SI{60}{\hertz} (depending on acoustic travel time, decreasing with imaging depth) \cite{minin_ultrafast_2011}. 

% A new disruptive imaging technology that has been long studied in research, but only recently made its entrance in clinical applications due to recent developments in graphical processing units (GPU), is ultrafast ultrasound. Instead of using multiple focused waves like in conventional ultrasound to construct an image, it uses a single plane wave to insonify the complete object to image \cite{tanter_ultrafast_2014}. Next, the image can be reconstructed from the recorded backscattered echoes by performing digital parallel beamforming. This dramatically increases the frame rate, allowing imaging at framerates of up to \SI{15000}{\hertz} (for superficial tissue, framerate drops with imaging depth) \cite{ minin_ultrafast_2011}. These ultrafast frame rates, open up new possibilities in e.g. the visualisation of transient shear waves, allowing the field of elastography to quantitatively determine the viscoelastic properties of tissue \cite{gennisson_ultrasound_2013}. 









% =======================================
% Introduce relevance of doing system identification of human joint dynamics. 
% -   Understanding human balance
% -   Identify the cause of movement disorders, to develop specialised treatment. How many people affected by movement disorders, and inaccurate treatment
% -   

% Many experiments have been conducted to understand the human neuromuscular system during various tasks (locomotion, balance control). There is a long sought explanation for the way the neuromuscular system works. (and how motor command are adapted during motion). 

% Various modelling attempts for locomotion and balance separate. 

% Neuromechanical models for locomotion focus more on the mechanical side, balance more on the neural side (small displacements) 


% By doing carefully designed experiments, a mathematical model can be fitted to the measured data. Having 
% system identification, doing experiments and fit model. 

% understand effects of neuromuscular deficits, predict results of rehabilitation. 

% Having an accurate mathematical model of neuromechanics can help ... 
% - determine the cause of a neuromuscular disorder - knowing the cause --> develop specialised treatment
% - development of active prosthetics \cite{markowitz_speed_2011}
% - sports, improving perfomance, preventing injuries


% \citeauthor{markowitz_speed_2011} developed a controller based on a fitted neuromuscular model emulating muscle reflexes to control an active speed adaptive ankle-foot prosthesis \cite{markowitz_speed_2011}. 

% Better understanding of the neuromuscular system can help in identifying the cause of movement disorders. Specialised treatment can be developed to help patients during rehabilitation to regain some degree or even full motor control. 


% % Determining the cause of a movement disorder hard: human adaptability after function loss. 

% identifying reflexes ... intrinsic and reflexive contributions...

% neuromuscular modelling hard because of: muscle redundancy, afferent signals cannot easily be measured. 


% % =======================================

% Because of these difficulties, there still is no consensus regarding the neuromuscular systems underlying standing balance. The contribution of the proprioceptive feedback from the different organs (i.e. muscle spindles and Golgi tendon organs) cannot be easily separated, because of many nested feedback loops. 

% Missing information, muscle tendon interaction (Discuss attempts Sinkjear and Loram…).
% \begin{itemize}
%     \setlength\itemsep{0em}
%     \item Attempts in acquiring this data, in vivo experiments hard to do...
%     \item Muscle activation (EMG) available, but this information lacks in… (?)
% \end{itemize}


% Need for measuring tendon and muscle length during experiments, various imaging techniques (MRI, ultrasound) available to do so, but temporal resolution too low in these conventional clinical imaging techniques. Processes in motor control very fast, reflexes happen within a couple milliseconds. Therefore, there is a need for high frequent measurement of what is happening in the muscle and/or tendon. 

% % ========================================
% % possible solution
% Recently, developments in processing power (GPU/CPU, and data transfer) made the development of ultrafast (plane wave) ultrasound systems possible. With this technique, the imaging frame rate is only limited by the time of flight of the ultrasonic waves through the tissue. Muscles and tendons can be imaged with this technique, allowing high frequent (> 1000 Hz) imaging, and thus providing high speed recordings of muscle-tendon dynamics. 

% Furthermore, ultrafast ultrasound has made it possible to visualise shear-waves in soft tissue, making it usable for elastography. By first inducing a focused shear wave, and then changing to high frequent imaging, the propagation of this shear-wave can be followed, from which the elastic properties of the tissue can be determined. 

% The high frame rate at which ultrafast ultrasound works, allows doing new noninvasive measurements. This gives rise to the question: To what extend can this ultrafast imaging technique provide more useful information in system identification of joint dynamics? What data can be measured during an identification process, and how can this data be utilised to improve system identification experiments? 
