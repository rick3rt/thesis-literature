
%\newcolumntype{Y}[1]{>{\footnotesize \hangindent=1em \raggedright \let\newline\\\arraybackslash}p{#1}}
\newcolumntype{Y}[1]{%
	>{\footnotesize\everypar{\hangindent=1em}\arraybackslash}p{#1}%
}
\renewcommand{\arraystretch}{1.2}
\newcommand{\customnewline}{\par}
% INSERT IN NEW TABLE
% \begin{tabular}{Y{7em}Y{5em}Y{10em}Y{10em}Y{10em}Y{10em}Y{10em}}
% \normalsize\textbf{Study}  & \normalsize\textbf{Joint}  & \normalsize\textbf{Limb structure} & \normalsize\textbf{Muscle-tendon unit interaction} & \normalsize\textbf{Muscle activation} & \normalsize\textbf{Afferent feedback} & \normalsize\textbf{Intrinsic feedback} \\





% Table generated by Excel2LaTeX from sheet 'Sheet1'
%\begin{table}[htbp]
%	\centering
%	\caption{Overview of the most important assumptions in the considered studies. \tred[Todo: add other perturbation signals]}
%	%   RESIZE IF NEEDED:
%	\resizebox{.95\linewidth}{!}{
%		\begin{tabular}{Y{7em}Y{10em}Y{10em}Y{10em}Y{10em}Y{10em}Y{10em}}
%			\toprule
%			\textbf{Study}  & \textbf{Joint and limb structure} & \textbf{Muscle-tendon unit interaction} & \textbf{Muscle activation} & \textbf{Perturbation signal} & \textbf{Afferent feedback} & \textbf{Intrinsic feedback} \\
%			\midrule
%			\citeauthor{zhang_simultaneous_1997} (\citeyear{zhang_simultaneous_1997}) \cite{zhang_simultaneous_1997} & Elbow joint, agonist and antagonist muscles lumped as one & None, measured angle proportional to muscle elongation & None, incorporated in time delays afferent feedback & \multicolumn{1}{l}{} & MS feedback, asymmetric velocity feedback, symmetric position feedback, single time delay.  \customnewline GTO feedback, gain with time delay, later omitted for simplification & \nth{2} order spring-damper system from torque to position, position fed to afferent feedback and inertia \\
%			\citeauthor{mirbagheri_intrinsic_2000} (\citeyear{mirbagheri_intrinsic_2000}) \cite{mirbagheri_intrinsic_2000} & Ankle joint, only active contribution plantar flexor & None, measured angle proportional to muscle elongation & \nth{3} order low-pass filter & \multicolumn{1}{l}{} & Velocity feedback, asymmetric, responsive to positive velocity (elongation plantar flexor muscle) & \nth{2} order spring-damper system, completely parallel to afferent pathway. Afferent contribution not affected by intrinsic properties \\
%			\citeauthor{van_der_helm_identification_2002} (\citeyear{van_der_helm_identification_2002}) \cite{van_der_helm_identification_2002} & Shoulder joint, shoulder and arm muscles (both agonist and antagonist) lumped as one & None, measured position proportional to muscle elongation & \nth{1} order low-pass filter & \multicolumn{1}{l}{} & MS feedback, position gain and velocity gain (linear, symmetric) & \nth{2} order spring-damper system, difference in handle and reflexive force affected by intrinsic parameters \\
%			\citeauthor{schouten_nmclab_2008} (\citeyear{schouten_nmclab_2008}) \cite{schouten_nmclab_2008} & Shoulder joint, shoulder and arm muscles (both agonist and antagonist) lumped as one & Tendon stiffness included, (lumped) tendon takes up part of movement, resulting in length contractile element & \nth{2} order low-pass filter & \multicolumn{1}{l}{} & MS feedback, position gain, velocity gain and acceleration gain (linear, symmetric) \customnewline GTO feedback, muscle force gain (linear, symmetric) & Separate inertia and properties emerging from co-contraction (stiffness, damping) \\
%			\citeauthor{mugge_rigorous_2010} (\citeyear{mugge_rigorous_2010}) \cite{mugge_rigorous_2010} & Ankle joint, agonist and antagonist muscles lumped as one & Tendon stiffness included, (lumped) tendon takes up part of movement, resulting in length contractile element & \nth{2} order low-pass filter & \multicolumn{1}{l}{} & MS feedback, position gain and velocity gain (linear, symmetric) \customnewline GTO feedback, muscle force gain (linear, symmetric) & Separate inertia and properties emerging from co-contraction (stiffness, damping) \\
%			\citeauthor{de_gooijer-van_de_groep_estimation_2016} (\citeyear{de_gooijer-van_de_groep_estimation_2016}) \cite{de_gooijer-van_de_groep_estimation_2016} & Wrist joint, antagonistic muscle model, flexor and extensor groups, introduced moment arms & None, moment arm varied with wrist angle & \nth{2} order low-pass filter, relating measured EMG to muscle active state & Ramp and Hold, due to nature perturbation, only active reflexive muscle force expected, no voluntary contraction expected (subjects asked to relax) & No proprioception modelled, reflexive force dependent on active state (from EMG) & Inertia of wrist, hand and manipulator handle, passive muscle tissue stiffness with relaxation (first order filter) \\
%			\bottomrule
%		\end{tabular}%
%	}
%	\label{tab:addlabel}%
%\end{table}%


% Table generated by Excel2LaTeX from sheet 'Sheet1'
\begin{table}[htbp]
%  \centering
  \caption{Overview of the most important assumptions in the considered studies.}
  %   RESIZE IF NEEDED:
 	\resizebox{.9\linewidth}{!}{
 	\begin{tabular}{Y{7em}Y{10em}Y{10em}Y{10em}Y{14em}Y{11em}Y{11em}}
    \toprule
    \textbf{Study}  & \textbf{Lumping structure} & \textbf{Muscle-tendon unit interaction} & \textbf{Muscle activation} & \textbf{(non)linear modelling} & \textbf{Afferent feedback} & \textbf{Intrinsic feedback} \\
    \midrule
    \citeauthor{zhang_simultaneous_1997} (\citeyear{zhang_simultaneous_1997}) \cite{zhang_simultaneous_1997} & Elbow joint, agonist and antagonist muscles lumped as one & None, measured angle proportional to muscle elongation & None, incorporated in time delays afferent feedback & Nonlinear (partially), time invariant model \customnewline White noise, low pass filtered, standard deviation \SI{1.5}{\deg} & MS feedback, asymmetric velocity feedback, symmetric position feedback, single time delay. \customnewline GTO feedback, gain with time delay, later omitted for simplification & \nth{2} order spring-damper system from torque to position, position fed to afferent feedback and inertia \\
    \citeauthor{mirbagheri_intrinsic_2000} (\citeyear{mirbagheri_intrinsic_2000}) \cite{mirbagheri_intrinsic_2000} & Ankle joint, only active contribution plantar flexor & None, measured angle proportional to muscle elongation & \nth{3} order low-pass filter & Nonlinear (partially), time invariant model \customnewline PRBS, peak-to-peak amplitude \SI{1.7}{\deg}, switching rate \SI{150}{\milli\second} & Velocity feedback, asymmetric, responsive to only positive velocity (elongation plantar flexor muscle) & \nth{2} order spring-damper system, completely parallel to afferent pathway. Afferent contribution not affected by intrinsic properties (i.e. inertia) \\
    \citeauthor{van_der_helm_identification_2002} (\citeyear{van_der_helm_identification_2002}) \cite{van_der_helm_identification_2002} & Shoulder joint, shoulder and arm muscles (both agonist and antagonist) lumped as one & None, measured position proportional to muscle elongation & \nth{1} order low-pass filter & Linear, time invariant model \customnewline Random continuous force perturbations, equal power at all frequencies, one wide band and two types of narrow band signals, max amplitude deviation wrist \SI{4.6}{\milli\meter} & MS feedback, position gain and velocity gain (linear, symmetric) & \nth{2} order spring-damper system, difference in handle and reflexive force affected by intrinsic parameters \\
    \citeauthor{schouten_nmclab_2008} (\citeyear{schouten_nmclab_2008}) \cite{schouten_nmclab_2008} & Shoulder joint, shoulder and arm muscles (both agonist and antagonist) lumped as one & Tendon stiffness included, (lumped) tendon takes up part of movement, resulting in length contractile element & \nth{2} order low-pass filter & Linear, time invariant model \customnewline Measured data of \citet{van_der_helm_identification_2002}  & MS feedback, position gain, velocity gain and acceleration gain (linear, symmetric)\customnewline GTO feedback, muscle force gain (linear, symmetric) & Separate inertia and properties emerging from co-contraction (stiffness, damping) \\
    \citeauthor{mugge_rigorous_2010} (\citeyear{mugge_rigorous_2010}) \cite{mugge_rigorous_2010} & Ankle joint, agonist and antagonist muscles lumped as one & Tendon stiffness included, (lumped) tendon takes up part of movement, resulting in length contractile element & \nth{2} order low-pass filter & Linear, time invariant model \customnewline Random perturbations, rectangular spectra, three different bandwidths (0.1 to 0.7, 1.2 and 2.0 Hz), supplemented with low power up to 40 Hz, max reported ankle deviation \SI{2}{\deg} & MS feedback, position gain and velocity gain (linear, symmetric)\customnewline GTO feedback, muscle force gain (linear, symmetric) & Separate inertia and properties emerging from co-contraction (stiffness, damping) \\
    \citeauthor{de_gooijer-van_de_groep_estimation_2016} (\citeyear{de_gooijer-van_de_groep_estimation_2016}) \cite{de_gooijer-van_de_groep_estimation_2016} & Wrist joint, antagonistic muscle model, flexor and extensor groups, introduced moment arms & None, moment arm dependent on wrist angle & \nth{2} order low-pass filter, relating measured EMG to muscle active state & Nonlinear, time invariant model \customnewline Ramp and Hold, due to nature perturbation, only active reflexive muscle force expected, no voluntary contraction expected (subjects asked to relax) & No proprioception modelled, reflexive force dependent on active state (from EMG) & Inertia of wrist, hand and manipulator handle, passive muscle tissue stiffness with relaxation (first order filter) \\
    \bottomrule
    \end{tabular}%
	}
  \label{tab:overview_assumptions}%
\end{table}%
